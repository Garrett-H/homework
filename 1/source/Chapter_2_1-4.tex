\documentclass[12pt]{article}

%Packages add more power to LaTeX documents
\usepackage{fullpage} %Otherwise there will be a lot of wasted space at the margins
\usepackage{enumerate} %For the multi-part problem in example #4
\usepackage{amsthm} %For proof environment
\usepackage{amsmath} %For math symbols (like the black square)
\usepackage{graphicx,float,wrapfig} %Including graphics like PDFs and some image formats.
\usepackage{algorithm}
\usepackage{algorithmic}

\author{Garrett Hay}
\title{CSCI 430: Homework 1}


\begin{document}
\maketitle

\section*{2.1-1}
\noindent    
\begin{tabular}{| c | c | c | c | c | c |}
      \hline
      31 & 41 & 59 & 26 & 41 & 58 \\
      \hline
      31 & 41 & 59 & 26 & 41 & 58 \\
      \hline
      31 & 41 & 59 & 26 & 41 & 58 \\
      \hline
      26 & 31 & 41 & 59 & 41 & 58 \\
      \hline
      26 & 31 & 41 & 41 & 59 & 58 \\
      \hline
      26 & 31 & 41 & 41 & 58 & 59 \\
      \hline
\end{tabular}
    
\section*{2.1-2}
\begin{algorithm}
\caption{Reverse Sort Sort}
\begin{algorithmic}
\FOR{$ j =  2 $ to $ A.length $}
	\STATE $ key = A[j] $
   	\STATE $ i = j-1 $
   	\WHILE{$ i > 0$ and $A[i] < key $}
   		\STATE $ A[i+1] = A[i] $
   		\STATE $ i = i-1 $
   	\ENDWHILE
   	\STATE $ A[i+1] = key $
\ENDFOR
\end{algorithmic}
\end{algorithm}

\newpage
\section*{2.1-3}
\noindent
\begin{algorithm}
\caption{Search for v}
\begin{algorithmic}
\STATE $locate = 0$
\FOR{$i=1$ to $A.length$}
   	\IF{$A[i] == v$}
   		\STATE $locate = i$
   	\ENDIF
\ENDFOR
\IF{$locate == 0$}
   	\STATE $print "NIL"$
\ELSE
   	\STATE $print(locate)$
\ENDIF
\end{algorithmic}
\end{algorithm}
	\subsection*{Initialization:}
    Show that 'locate', the variable that will hold the location of 'v' in the array, is 0. A number nonexistent in A[1...n].
    \subsection*{Maintenance:}
    Show that the loop maintains. The body of the loop checks if A[1], A[2], and so on by 1 position to see if the current element is the same as the value of 'v' until A[n]. When an element satisfies the condition if changes 'locate' to the location in the array.
    \subsection*{Termination:}
    Condition of for loop termination is that $i > A.length$ or n. Because each iterator increase by 1, we must have $i=(n+1)$ at that time. Substituting $i$ for $(n+1)$ in the loop we have either 'locate' is 0 or the $i$ were $A[i]$ is equal to $v$ in $A[1-n]$. Then the if-else statement decides if $locate$ is 0 to print "NIL" or to print the location of $v$ in A[]. Hence the algorithm is correct.
\section{2.1-4}
    \subsection*{Formal:}
    \textbf{Input}: 2 binary numbers, A and B, represented by A[0...n] and B[0...n] in binary form\\
    \textbf{Output}: Binary number $C[0...(N+1)]$ where $C = A+B$.
    \subsection*{Pseudo:}   
	\begin{algorithm}
	\caption{Binary Addition}
	\begin{algorithmic} 
	\STATE $carry = 0$
	\STATE $ i = A.length $
	\WHILE{$i < 0$}
		\STATE $added = A[i] + B[i] +carry $
		\STATE $C[i + 1] = added \% 2 $
		\STATE $ carry = added \div 2 $
		\STATE $ i = i - 1$
	\ENDWHILE
	\STATE $ C[0] = carry$
	\end{algorithmic}
	\end{algorithm}
\end{document}