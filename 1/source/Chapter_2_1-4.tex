\documentclass[12pt]{article}

%Packages add more power to LaTeX documents
\usepackage{fullpage} %Otherwise there will be a lot of wasted space at the margins
\usepackage{enumerate} %For the multi-part problem in example #4
\usepackage{amsthm} %For proof environment
\usepackage{amsmath} %For math symbols (like the black square)
\usepackage{graphicx,float,wrapfig} %Including graphics like PDFs and some image formats.

\author{Garrett Hay}
\title{CSCI 430: Homework 1}


\begin{document}
\maketitle

\section*{2.1-1}
\noindent

%Number \item tags
\begin{enumerate}
  %Propositional logic examples
  % The $$ symbols are used to specify mathematics inline.
  \item $ \lnot (p \land q) \rightarrow (\top \lor \bot) $

    %First-order logic examples
  \item  $\forall$ x $\in$ X, $\exists$ y$\in$ Y s.t. A(x) $\rightarrow$ B(y)
    
  \item A truth table:
    
    \begin{tabular}{| c | c | c | c | c | c |} %Specifies a table with three centered columns
      \hline %horizontal line
      31 & 41 & 59 & 26 & 41 & 58 \\
      31 & 41 & 59 & 26 & 41 & 58 \\
      31 & 41 & 59 & 26 & 41 & 58 \\
      26 & 31 & 41 & 59 & 41 & 58 \\
      26 & 31 & 41 & 41 & 59 & 58 \\
      26 & 31 & 41 & 41 & 58 & 59 \\
      \hline
    \end{tabular}
    
    \section*{2.1-2}
    \noindent
    For j=2 to A.length\\
    	key = A[j]\\
    	i = j-1\\
    	while (i>0) && (A[i]<key)\\
     		A[i+1] = A[i]\\
     		i = i-1\\
     	A[i+1] = key\\
    \section{2.1-3}
    \noindent
    locate = 0\\
    For i=1 to A.length\\
    	if(A[i] == v)\\
    		locate = i\\
    if{locate == 0)\\
    	print "NIL"\\
    else:\\
    	print(locate)\\
    \\
    \subsection*{Initialization:}
    Show that 'locate', the variable that will hold the location of 'v' in the array, is 0. A number nonexistent in A[1...n].
    \subsection*{Maintenance:}
    Show that the loop maintains. The body of the loop checks if A[1], A[2], and so on by 1 position to see if the current element is the same as the value of 'v' until A[n]. When an element satisfies the condition if changes 'locate' to the location in the array.
    \subsection*{Termination:}
    Condition of for loop termination is that 'i' > A.length/n. Because each iterator increase by 1, we must have i=(n+1) at that time. Substituting 'i' for (n+1) in the loop we have either 'locate' is 0 or the 'i' were A[i] is equal to 'v' in A[1-n]. Then the if-else statement decides if 'locate' is 0 to print "NIL" or to print the location of v in A[]. Hence the algorithm is correct.
    \section{2.1-4}
    \subsection*{Formal:}
    Input: 2 binary numbers, A and B, represented by A[0...n] and B[0...n] in binary form\\
    Output: Binary number C[0...(N+1)] where C = A+B.
    \subsection*{Pseudo:}
    int carry = 0\\
    for( i=A.length to 0); (-=1)\\
    	added = A[i] + B[i] + carry\\
    	C[i+1] = added\%2\\
    	carry = added/2\\
    C[0] = carry\\
    % The \\ signals a newline.
  %\item The following is a finite state automaton typeset with graphviz:\\
    %\includegraphics[width=10cm]{fsa_even.pdf}
    
\end{enumerate}
    
\end{document}